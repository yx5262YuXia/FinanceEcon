% Name: Yx5262 Problem Set 9 Yu Xia v0.tex
% Author: Yu Xia 
% Description: The version before submitted (to be cont.)
% Last Updated: Nov 10, 2022
\documentclass{article}

\usepackage{titlesec}

\usepackage{amsmath}
\usepackage{amssymb}

\usepackage{booktabs}
\usepackage{float}
\usepackage{colortbl}
\usepackage{xcolor}

\usepackage{a4wide}
\usepackage{setspace}
\usepackage{geometry}
\usepackage{parskip}

\usepackage{multirow}
\usepackage{adjustbox}
\usepackage{graphicx}

\usepackage{mathtools}
\allowdisplaybreaks

\usepackage{hyperref}
\hypersetup{
    colorlinks=true,
    linkcolor=black,
    urlcolor=blue
}

\DeclareRobustCommand{\bbone}{\text{\usefont{U}{bbold}{m}{n}1}}

\titleformat*{\subsection}{\normalfont}

\author{Yu Xia \\ ID: yx5262}
\title{Yu Xia's Answer for Problem Set 9}
\date{Fall 2022}

\begin{document}
\maketitle

\nocite{*}

\section*{1}

\subsection*{(a)}

Denote the risk-neutral probability measure as $\left\{\pi, 1-\pi\right\}$.

By the non-arbitrage pricing formula:

$\quad 100=\dfrac{120\pi+80\left(1-\pi\right)}{1.04}$

$\iff$

$\quad 104=120\pi+80-80\pi$

$\iff$

$\quad 24=40\pi$

$\iff$

$\quad \pi=\dfrac{24}{40}=\dfrac{8\times3}{8\times5}=0.6$

Check other cases:

$\quad 120=\dfrac{144\pi+96\left(1-\pi\right)}{1.04} \iff \pi=0.6$

$\quad 80=\dfrac{96\pi+64\left(1-\pi\right)}{1.04} \iff \pi=0.6$

which is reasonable because the price change factor is $u=1.2$ when price goes up and $d=0.8$ when price goes down. It holds for all 3 periods.

\subsection*{(b)}

$\because K\in \left(80, 96\right),$

$\therefore$ The option will be exercised only if price never goes up. 

The value of exercise at $t=2$ is:

$\quad\dfrac{\left(1-\pi\right)^{2}}{\left(1+r\right)^{2}}\left(K-64\right)=\dfrac{0.4^{2}}{1.04^{2}}\left(K-64\right)$

The value of exercise at $t=1$ is:

$\quad\dfrac{1-\pi}{1+r}\left(K-80\right)=\dfrac{0.4}{1.04}\left(K-80\right)$

The value of exercise at $t=1$ is: $K-100$

American put option will be exercised prematurely iff:

$\quad\dfrac{0.4^{2}}{1.04^{2}}\left(K-64\right)<\min \left\{ K-100, \dfrac{0.4}{1.04}\left(K-80\right) \right\}$

When $\dfrac{0.4^{2}}{1.04^{2}}\left(K-64\right)<K-100$

$\iff$

$\quad \dfrac{0.4^{2}}{1.04^{2}}K-\dfrac{0.4^{2}}{1.04^{2}}\times64<K-100$

$\iff$

$\quad 100-\dfrac{0.4^{2}}{1.04^{2}}\times64<K-\dfrac{0.4^{2}}{1.04^{2}}K$

$\iff$

$\quad \left(1-\dfrac{0.4^{2}}{1.04^{2}}\right)K>100-\dfrac{0.4^{2}}{1.04^{2}}\times64$

$\iff$

$\quad K>\dfrac{100-\dfrac{0.4^{2}}{1.04^{2}}\times64}{1-\dfrac{0.4^{2}}{1.04^{2}}}=106.25$

When $\dfrac{0.4^{2}}{1.04^{2}}\left(K-64\right)<\dfrac{0.4}{1.04}\left(K-80\right)$

$\iff$

$\quad\dfrac{0.4}{1.04}\left(K-64\right)<K-80$

$\iff$

$\quad\dfrac{0.4}{1.04}K-\dfrac{0.4}{1.04}\times64<K-80$

$\quad80-\dfrac{0.4}{1.04}\times64<K-\dfrac{0.4}{1.04}K$

$\quad\dfrac{1}{1.04}K>80-\dfrac{0.4}{1.04}\times64$

$\quad K>1.04\times80-0.4\times64=90$

So American put option will be exercise prematurely iff $K\in\left(90,96\right)$

\subsection*{(c)}

We have:

$P_{0}\left(K\right)=\begin{dcases}
    \dfrac{0.4^{2}}{1.04^{2}}\left(K-64\right) & 80<K<90, \\
    \dfrac{0.4}{1.04}\left(K-80\right) & 90\leqslant K<96.\\
\end{dcases}$

\section*{2}

In high price level case $\omega_{1}$, the operational profit each year $T$ is:

$\quad \Pi_{T}\left(\omega_{1}\right)=5\cdot50-150=100$

In low price level case $\omega_{2}$:

$\quad \Pi_{T}\left(\omega_{2}\right)=5\cdot30-150=0$

Thus

$\quad \displaystyle V\left(\omega_{i}\right)=\sum^{\infty}_{t=1}\beta^{t}\Pi_{t}\left(\omega_{i}\right)=\dfrac{\Pi_{T}\left(\omega_{i}\right)}{r}=10\Pi_{T}\left(\omega_{i}\right)$

We have

$\quad V\left(\omega_{1}\right)=10\Pi_{T}\left(\omega_{1}\right)=1000$

$\quad V\left(\omega_{2}\right)=10\Pi_{T}\left(\omega_{2}\right)=0$

The $PV$ of future profit is:

$\quad V_{0}=0.6\cdot1000+0.4\cdot0=600$

$\quad NPV_{0;0}=\max\left\{V_{0}-I, 0\right\}=\left(600-I\right)_{+}$

Instead, assume that the manager waits one period. Then, if price goes up, the $PV$ of profit (without accounting for fixed investment cost $I$) is:

$\quad \displaystyle \sum^{\infty}_{t=1}\beta^{t}\Pi_{t}\left(\omega_{1}\right)=1000$

Therefore,

$\quad NPV_{1;1}\left(\omega_{1}\right)=\left(1000-I\right)_{+}$

Similarly,

$\quad NPV_{1;1}\left(\omega_{2}\right)=\left(-I\right)_{+}$

Hence,

$\quad\displaystyle NPV_{0;1}=\mathbb{E} ^{\mathbb{Q} }\left[\beta NPV_{1;1}\right]=\beta\sum_{\omega\in\Omega}p\left(\omega\right)NPV_{1;\geqslant1}\left(\omega\right)=\dfrac{1}{1.1}\left(0.6\left(1000-I\right)_{+}+0.4\left(-I\right)_{+}\right)$

$\quad V_{\text{opt}}=NPV_{0;1}-NPV_{0;0}=\dfrac{1}{1.1}\left(0.6\left(1000-I\right)_{+}+0.4\left(-I\right)_{+}\right)-\left(600-I\right)_{+}$

\subsection*{(a)}

Never invest if: $\left(600-I\right)_{+}=\left(1000-I\right)_{+}=\left(-I\right)_{+}=0$

$\quad I>1000$

If $I=1000$, it would be indifferent to whether or not to invest in the highest price case. 

\subsection*{(b)}

Invest at $t=1$ is optimal if: $\max\left\{\left(1000-I\right)_{+},\left(-I\right)_{+}\right\}>0$ and $V_{\text{opt}}>0$.

That is:

$\quad I<1000$ and $\dfrac{1}{1.1}\left(0.6\left(1000-I\right)_{+}+0.4\left(-I\right)_{+}\right)-\left(600-I\right)_{+}>0$

$\iff$

$\quad \left(600-I\right)_{+}<\dfrac{1}{1.1}\left(600-0.6I+0.4\left(-I\right)_{+}\right)$

If $1000>I\geqslant600$, $\left(600-I\right)_{+}=0$, $\left(-I\right)_{+}=0$, $\left(1000-I\right)_{+}>0$ holds.

The firm will invest at $t=1$ rather than $t=0$. It is reasonable because the firm will wait for a period, and invest when the price goes up. 

If $0<I\leqslant600$, $\left(600-I\right)_{+}=600-I$, $\left(1000-I\right)_{+}=1000-I$, $\left(-I\right)_{+}=0$

To satisfy $\left(600-I\right)_{+}<\dfrac{1}{1.1}\left(600-0.6I+0.4\left(-I\right)_{+}\right)$,

$\quad 600-I<\dfrac{1}{1.1}\left(600-0.6I\right)$

$\quad 1.1\left(600-I\right)<600-0.6I$

$\quad 660-1.1I<600-0.6I$

$\quad 600-0.6I>660-1.1I$

$\quad 1.1I+600-0.6I>660$

$\quad 1.1I-0.6I>60$

$\quad 0.5I>60$

$\quad I>120$

If $I=120$, the investor will be indifferent with investing at $t=0$ and wait for the price to go up for a period to invest.

In conclusion, if $120<I<1000$, invest at time $t=1$.

\subsection*{(c)}

Invest at $t=0$ if: $\left(600-I\right)_{+}>0$ and $V_{\text{opt}}<0$

That is,

$\quad I<600$, and $\quad \dfrac{1}{1.1}\left(0.6\left(1000-I\right)_{+}+0.4\left(-I\right)_{+}\right)-\left(600-I\right)_{+}<0$

$\quad 600-I>\dfrac{1}{1.1}\left(0.6\left(1000-I\right)+0.4\left(-I\right)_{+}\right)$

$\quad 1.1\left(600-I\right)>\left(600-0.6I\right)+0.4\left(-I\right)_{+}$

$\quad 660-1.1I>600-0.6I+0.4\left(-I\right)_{+}$

$\quad 60-0.5I>0.4\left(-I\right)_{+}$

$\left(-I\right)_{+}=0$ if $I>0$, which is usually the case.

If a negative $I$ is allowed, 

$\quad 60-0.5I>-0.4I$

$\quad 60-0.1I>0$

$\quad 0.1I<60$

$\quad I<600$

So invest at time $t=0$ if $I<120$. (If a negative $I$ is not allowed, $0<I<120$.)

\section*{3}

Time 2 $PV$  of the future profit:

If price goes up:

$\quad\displaystyle \sum^{\infty}_{s=1}\beta^{s}\left(R_{2}\left(40\right)-C\right)=\dfrac{40\cdot5-150}{r}=\dfrac{200-150}{0.1}=500$

$\quad V_{2}\left(40\right)=500$ if $50+\dfrac{Sc^{h}}{1.1}<500$

If $50+\dfrac{Sc^{h}}{1.1}>500$:

$\quad \dfrac{Sc^{h}}{1.1}>450$

$\iff$

$\quad Sc^{h}>450\cdot1.1=495$

$\quad V_{2}\left(40\right)=50+\dfrac{Sc^{h}}{1.1}$ if $Sc^{h}>495$

The firm would be indifferent if $Sc^{h}=495$.

If price falls:

$\quad \Pi_{2}\left(20\right)=20\cdot5-150=-50<0$

$\quad V_{2}\left(20\right)=-50+\dfrac{Sc^{l}}{1.1}$

Usually $Sc^{l}>0$. It is optimal to exit as well if: 

$\quad -50+\dfrac{Sc^{l}}{1.1}>-500$

$\quad \dfrac{Sc^{l}}{1.1}>-450$

$\quad Sc^{l}>-450\cdot1.1=-495$

Value of staying active at $t=1$:

$\quad V_{1}\left(S_{1}\right)=V_{1}\left(30\right)=\dfrac{1}{1.1}\left(0.6V_{2}\left(40\right)+0.4V_{2}\left(20\right)\right)$

Value of exit at $t=1$:

$\quad NPV_{1}\left(S_{1}\right)=NPV_{1}\left(30\right)=\beta Sc=\dfrac{121}{1.1}=110$

\subsection*{(a)}

Equation holds when the firm is indifferent.

If exit at $t=1$ is better than staying in $t=1$ in terms of expectation:

$\quad V_{1}\left(S_{1}\right)<NPV_{1}\left(S_{1}\right)$

$\iff$

$\quad \dfrac{1}{1.1}\left(0.6V_{2}\left(40\right)+0.4V_{2}\left(20\right)\right)<\dfrac{Sc}{1.1}$

$\iff$

$\quad 0.6V_{2}\left(40\right)+0.4V_{2}\left(20\right)<Sc=121$

If $Sc^{h}>495$ and $Sc^{l}>-495$:

$\quad 0.6\left(50+\dfrac{Sc^{h}}{1.1}\right)+0.4\left(-50+\dfrac{Sc^{l}}{1.1}\right)<121$

$\iff$

$\quad 30+\dfrac{6}{11}Sc^{h}-20+\dfrac{4}{11}Sc^{l}<121$

$\iff$

$\quad 10+\dfrac{6}{11}Sc^{h}+\dfrac{4}{11}Sc^{l}<121$

$\iff$

$\quad \dfrac{6}{11}Sc^{h}+\dfrac{4}{11}Sc^{l}<111$

If $Sc^{h}<495$ and $Sc^{l}>-495$:

$\quad 0.6\cdot500+0.4\left(-50+\dfrac{Sc^{l}}{1.1}\right)<Sc=121$

$\iff$

$\quad 6\cdot50+\left(-0.4\cdot50+\dfrac{0.4}{1.1}Sc^{l}\right)<121$

$\iff$

$\quad 300+\left(-4\cdot5+\dfrac{4}{11}Sc^{l}\right)<121$

$\iff$

$\quad -20+\dfrac{4}{11}Sc^{l}<-179$

$\iff$

$\quad \dfrac{4}{11}Sc^{l}<-159$

$\iff$

$\quad Sc^{l}<-159\cdot\dfrac{11}{4}=-437.25$

If $Sc^{h}>495$ and $Sc^{l}<-495$:

$\quad 0.6\left(50+\dfrac{Sc^{h}}{1.1}\right)+0.4\left(-500\right)<Sc=121$

$\quad 30+\dfrac{6}{11}Sc^{h}-0.4\cdot500<121$

$\quad \dfrac{6}{11}Sc^{h}-4\cdot50<91$

$\quad \dfrac{6}{11}Sc^{h}<291$

$\quad Sc^{h}<291\cdot\dfrac{11}{6}=533.5$

If $Sc^{h}<495$ and $Sc^{l}<-495$:

$\quad 0.6V_{2}\left(40\right)+0.4V_{2}\left(20\right)=0.6\cdot500+0.4\cdot\left(-500\right)=0.2\cdot500=2\cdot50=100<121=Sc$

In conclusion, 

case 1:

$\quad Sc^{h}>Sc^{l}$, $Sc^{h}>495$, $Sc^{l}>-495$ and $\dfrac{6}{11}Sc^{h}+\dfrac{4}{11}Sc^{l}<111$,

case 2:

$\quad Sc^{h}>Sc^{l}$, $Sc^{h}<495$, $-495<Sc^{l}<-437.25$,

if a negative $Sc^{l}$ is alowed,

case 3:

$\quad Sc^{h}>Sc^{l}$, $495<Sc^{h}<533.5$ and $Sc^{l}<-495$,

case 4:

$\quad Sc^{h}>Sc^{l}$, $Sc^{h}<495$ and $Sc^{l}<-495$.

\subsection*{(b)}

If not exit in $t=1$:

$\quad Sc^{h}>Sc^{l}$, $Sc^{h}>495$, $Sc^{l}>-495$ and $\dfrac{6}{11}Sc^{h}+\dfrac{4}{11}Sc^{l}>111$

or 

$\quad Sc^{h}>Sc^{l}$, $Sc^{h}<495$ and $Sc^{l}>-437.25$

or if negative values are allowed,

$\quad Sc^{h}>Sc^{l}$, $Sc^{h}>533.5$ and $Sc^{l}<-495$

Quit if price goes down:

$\quad Sc^{l}>-495$

Quit if price goes up:

$\quad Sc^{h}>495$

Quit if no matter what happens:

$\quad Sc^{h}>495$ and $Sc^{l}>-495$ as well.

\section*{4}

Assume the firm is active at $t=2$.

Let $S_{2}=50$.

The $PV$ of the profit is: 

$\quad \dfrac{50\cdot5-150}{0.1}=\dfrac{250-150}{0.1}=\dfrac{100}{0.1}=1000>400=Sc$

Not optimal to quit.

$\quad V^{\text{ac}}_{2}\left(50\right)=1000$

Let $S_{2}=40$.

The $PV$ of the profit is: 

$\quad \dfrac{40\cdot5-150}{0.1}=\dfrac{200-150}{0.1}=\dfrac{50}{0.1}=500>400=Sc$

Not optimal to quit.

$\quad V^{\text{ac}}_{2}\left(40\right)=500$

Let $S_{2}=20$.

The $PV$ of the profit at this period is: 

$\quad 20\cdot5-150=100-150=-50=-50<0$

The firm will exit if the price goes down to 20. 

$\quad V^{\text{ac}}_{2}\left(20\right)=-50+\dfrac{400}{1.1}\approx313.64$

Now look at the firm's decision at $t=1$:

Let $S_{1}=50$.

The $PV$ of the profit is:

$\quad \dfrac{50\cdot5-150}{0.1}=\dfrac{250-150}{0.1}=\dfrac{100}{0.1}=1000>400=Sc$

Let $S_{1}=30$, the $EPV$ of the profit is:

$\quad \mathbb{E} ^{\mathbb{Q} }_{1}\left[\beta V^{\text{ac}}_{2}\right]=\mathbb{E} ^{\mathbb{Q} }\left[\beta V^{\text{ac}}_{2}|S_{1}=40\right]\approx0.6\cdot500+0.4\cdot\dfrac{313.64}{1.1}\approx414.05$

If the firm, instead, decides to exit at $t=1$, when $S_{1}=50$, it will gain $\dfrac{Sc}{1.1}=\dfrac{400}{1.1}\approx363.64<414.05$.

It's not optimal to exit at $t=1$ if $S_{1}=30$.

To summarize:

$\quad V^{\text{ac}}_{1}\left(50\right)=1000$

$\quad V^{\text{ac}}_{1}\left(30\right)=414.05$

At $t=0$, the $EPV$ of future gains, is:

$V^{\text{ac}}_{0}=\mathbb{E} ^{\mathbb{Q} }\left[\beta V^{\text{ac}}_{1}\right]=0.6\cdot V^{\text{ac}}_{1}\left(50\right)+0.4\cdot V^{\text{ac}}_{1}\left(30\right)\approx0.6\cdot1000+0.4\cdot\dfrac{414.05}{1.1}\approx750.56$

Backward induction:

At $t=2$:

$\quad G_{2}\left(50\right)=V^{\text{ac}}_{2}\left(50\right)-I=1000-500=500$

$\quad G_{2}\left(40\right)=V^{\text{ac}}_{2}\left(40\right)-I=500-500=0$

$\quad G_{2}\left(20\right)=\left(V^{\text{ac}}_{2}\left(20\right)-I, 0\right)_{+}=\left(313.64-500, 0\right)_{+}=0$

Hence the firm invests at $t=2$ only if $S_{2}=50$ or $S_{2}=40$.

At $t=1$:

$\quad G_{1}\left(50\right)=V^{\text{ac}}_{1}\left(50\right)-I=1000-500=500>\dfrac{500}{1.1}=\dfrac{G_{2}}{1.1}$

$\quad G_{1}\left(30\right)=\left(V^{\text{ac}}_{1}\left(30\right)-I, 0\right)_{+}=\left(414.05-500, 0\right)_{+}=0$

Hence the firm invests at $t=1$ immediately only if $S_{1}=50$.

At $t=0$:

$\quad \beta\mathbb{E} ^{\mathbb{Q} }\left[G_{1}\right]=\dfrac{1}{1.1}\left(0.6G_{1}\left(50\right)+0.4G_{1}\left(30\right)\right)=\dfrac{1}{1.1}\left(0.6\cdot500+0\right)=\dfrac{6\cdot50}{1.1}=\dfrac{300}{1.1}\approx272.73$

$\quad V^{\text{ac}}_{0}-I\approx750.56-500\approx250.56<272.73$

In conclusion:

\subsection*{(a)}

Don't invest immediately at $t=0$.

If $S_{1}=30$, wait until $t=2$. 

Do not invest at $t=2$ if $S_{2}=20$ after waiting in the previous period.

\subsection*{(b)}

If $S_{1}=50$, the firm invests at $t=1$ immediately.

If $S_{1}=30$, wait until $t=2$. If $S_{2}=40$, invest immediately. 

\section*{5}

Assume the firm is active at $t=2$.

Let $S_{2}=60$.

The $PV$ of the profit is: 

$\quad \dfrac{60\cdot5-150}{0.1}=\dfrac{300-150}{0.1}=\dfrac{150}{0.1}=1500>200=Sc$

Not optimal to quit.

$\quad V^{\text{ac}}_{2}\left(60\right)=1500$

Let $S_{2}=40$.

The $PV$ of the profit is: 

$\quad \dfrac{40\cdot5-150}{0.1}=\dfrac{200-150}{0.1}=\dfrac{50}{0.1}=500>200=Sc$

Not optimal to quit.

$\quad V^{\text{ac}}_{2}\left(40\right)=500$

Let $S_{2}=20$.

The $PV$ of the profit at this period is: 

$\quad 20\cdot5-150=100-150=-50=-50<0$

The firm will exit if the price goes down to 20. 

$\quad V^{\text{ac}}_{2}\left(20\right)=-50+\dfrac{200}{1.1}\approx131.82$

Now look at the firm's decision at $t=1$:

Let $S_{1}=50$, the $EPV$ of the profit is:

$\quad \mathbb{E} ^{\mathbb{Q} }\left[\beta V^{\text{ac}}_{2}|S_{1}=50\right]\approx0.6\cdot1500+0.4\cdot500=6\cdot150+4\cdot50=3\cdot300+200=1100$

Let $S_{1}=30$, the $EPV$ of the profit is:

$\quad \mathbb{E} ^{\mathbb{Q} }\left[\beta V^{\text{ac}}_{2}|S_{1}=30\right]\approx0.6\cdot500+0.4\cdot\dfrac{131.82}{1.1}\approx347.93$

If the firm, instead, decides to exit at $t=1$, it will gain $\dfrac{Sc}{1.1}=\dfrac{200}{1.1}\approx181.82<347.93<1100$.

The firm will not exit at $t=1$ before observing the price at $t=2$.

To summarize:

$\quad V^{\text{ac}}_{1}\left(50\right)=1100$

$\quad V^{\text{ac}}_{1}\left(30\right)=347.93$

At $t=0$, the $EPV$ of future gains, is:

$V^{\text{ac}}_{0}=\mathbb{E} ^{\mathbb{Q} }\left[\beta V^{\text{ac}}_{1}\right]=0.6\cdot V^{\text{ac}}_{1}\left(50\right)+0.4\cdot V^{\text{ac}}_{1}\left(30\right)\approx0.6\cdot1100+0.4\cdot\dfrac{347.93}{1.1}\approx786.52$

Backward induction:

At $t=2$:

$\quad G_{2}\left(60\right)=V^{\text{ac}}_{2}\left(60\right)-I=1500-400=1100$

$\quad G_{2}\left(40\right)=V^{\text{ac}}_{2}\left(40\right)-I=500-400=100$

$\quad G_{2}\left(20\right)=\left(V^{\text{ac}}_{2}\left(20\right)-I, 0\right)_{+}=\left(131.82-400, 0\right)_{+}=0$

Hence the firm invests at $t=2$ only if $S_{2}=60$ or $S_{2}=40$.

At $t=1$:

$\quad G_{1}\left(50\right)=V^{\text{ac}}_{1}\left(50\right)-I=1100-400=700$

$\quad G_{1}\left(30\right)=\left(V^{\text{ac}}_{1}\left(30\right)-I, 0\right)_{+}=\left(347.93-400, 0\right)_{+}=0$

Hence the firm invests at $t=1$ immediately only if $S_{1}=50$.

At $t=0$:

$\quad \beta\mathbb{E} ^{\mathbb{Q} }\left[G_{1}\right]=\dfrac{1}{1.1}\left(0.6G_{1}\left(50\right)+0.4G_{1}\left(30\right)\right)=\dfrac{1}{1.1}\left(0.6\cdot700+0\right)=\dfrac{420}{1.1}\approx381.82$

$\quad V^{\text{ac}}_{0}-I\approx786.52-400\approx386.52>381.82$

In conclusion:

\subsection*{(a)}

If $S_{1}=30$, wait until $t=2$. 

Do not invest at $t=2$ if $S_{2}=20$ after waiting in the previous period.

\subsection*{(b)}

Invest at $t=0$.

If $S_{1}=50$, the firm invests at $t=1$ immediately.

If $S_{1}=40$, wait until $t=2$. If $S_{2}=40$, invest immediately. 

\end{document}