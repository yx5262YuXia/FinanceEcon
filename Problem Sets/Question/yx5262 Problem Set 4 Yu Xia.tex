\documentclass{article}

\usepackage{titlesec}

\usepackage{amsmath}
\usepackage{amssymb}

\usepackage{booktabs}
\usepackage{float}
\usepackage{colortbl}
\usepackage{xcolor}

\usepackage{a4wide}
\usepackage{setspace}
\usepackage{geometry}
\usepackage{parskip}

\usepackage{multirow}
\usepackage{adjustbox}
\usepackage{graphicx}

\usepackage{hyperref}
\hypersetup{
    colorlinks=true,
    linkcolor=black,
    urlcolor=blue
}

\DeclareRobustCommand{\bbone}{\text{\usefont{U}{bbold}{m}{n}1}}

\DeclareMathOperator{\EX}{\mathbb{E}}

\titleformat*{\subsection}{\normalfont}

\author{Yu Xia \\ ID: yx5262}
\title{Yu Xia's Problem Set 4}
\date{September 2022}

\begin{document}
\maketitle

\nocite{*}

\section*{1}

$r^{m}_{15}=\dfrac{r^{a}_{15}}{12}=0.003$

We have:

$$
M_{15}=\dfrac{P\cdot r^{m}_{15}}{1-\dfrac{1}{\left(1+r^{m}_{15}\right)^{180}}}
$$

If $P=\$160$K, 

$M_{15}=\boxed{1,151.69}$

$TP_{0}=180M_{15}=\boxed{207,303}$

If $P=\$150$K, 

$M_{15}=\boxed{1,079.71}$

$TP_{1}=180M_{15}=\boxed{194,346}$

In 5.1, $\rho^{a}=0.02$ is given.

$\rho^{m}=\dfrac{\rho^{a}}{12}=\dfrac{0.02}{12}=\dfrac{0.01}{6}=\dfrac{1}{600}\approx0.017$

The present value of the total annual rent is

$$
C\left(c\right)=c\left(1+\dfrac{1}{1+\rho^{m}}+\dots+\dfrac{1}{\left(1+\rho^{m}\right)^{11}}\right)
=c\dfrac{1+\rho^{m}}{\rho^{m}}\left(1-\dfrac{1}{\left(1+\rho^{m}\right)^{12}}\right)
$$

If $C=\$1,500$, $\boxed{C\left(1500\right)=17,836}$

$TP_{0}-C\left(c\right)-\dfrac{TP_{1}}{1+\rho^{a}}=207,303-17,836-\dfrac{194,346}{1.02}\approx189,467-190,535.29\approx\boxed{-1,068.29}$

If $C=\$2,000$, $\boxed{C\left(2000\right)=23,782}$

$TP_{0}-C\left(c\right)-\dfrac{TP_{1}}{1+\rho^{a}}=207,303-23,782-\dfrac{194,346}{1.02}\approx183,521-190,535.29\approx\boxed{-7,014.29}$

\begin{flalign*}
    c^{*}&=\dfrac{\rho^{m}}{1+\rho^{m}}\dfrac{TP_{0}-\dfrac{TP_{1}}{1+\rho^{a}}}{1-\dfrac{1}{\left(1+\rho_{m}\right)^{12}}}&\\
    &=\dfrac{\dfrac{1}{600}}{1+\dfrac{1}{600}}\dfrac{207,303-\dfrac{194,346}{1+0.02}}{1-\dfrac{1}{\left(1+\dfrac{1}{600}\right)^{12}}}&\\
    &=\dfrac{\dfrac{1}{600}}{\dfrac{601}{600}}\dfrac{207,303-\dfrac{194,346}{1.02}}{1-\dfrac{1}{\left(\dfrac{601}{600}\right)^{12}}}&\\
    &=\dfrac{600}{601}\dfrac{207,303-\dfrac{194,346}{1.02}}{1-\left(\dfrac{600}{601}\right)^{12}}&\\
    &\approx\boxed{1410.1423}
\end{flalign*}

\section*{2}

If $r^{a}_{u}=0.04$

$r^{m}_{u}=\dfrac{r^{a}_{u}}{12}=\dfrac{0.04}{12}=\dfrac{0.01}{3}=\dfrac{1}{300}\approx0.00333$

When $P=\$150$K,

$M_{u}=\dfrac{P\cdot r^{m}_{u}}{1-\dfrac{1}{\left(1+r^{m}_{u}\right)^{180}}}=\dfrac{150,000\cdot \dfrac{1}{300}}{1-\dfrac{1}{\left(\dfrac{301}{300}\right)^{180}}}=\dfrac{1500\cdot \dfrac{1}{3}}{1-\left(\dfrac{300}{301}\right)^{180}}\approx\dfrac{500}{1-0.5493595}\approx\dfrac{500}{0.45}\approx\boxed{1,109.53}$

$TP_{u}=180M_{u}\approx180\times1,109.53\approx199,715.7$

$\mathbb{E}\left[TP\right]=\pi TP_{u}+\left(1-\pi\right)TP_{1}\approx0.5\times199,715.7+0.5\times194,346\approx\boxed{197,030.9} $

\section*{3}

If the prices go up by 10 percent, $P_{u}=170$K.

$M_{u}=\dfrac{P_{u}\cdot r^{m}_{15}}{1-\dfrac{1}{\left(1+r^{m}_{15}\right)^{180}}}=\dfrac{170,000\cdot 0.003}{1-\dfrac{1}{\left(1+\dfrac{3}{1000}\right)^{180}}}=\dfrac{170\cdot 3}{1-\dfrac{1}{\left(\dfrac{1003}{1000}\right)^{180}}}=\dfrac{510}{1-\left(\dfrac{1000}{1003}\right)^{180}}\approx\boxed{1,223.6658}$

$TP_{u}=180M_{u}\approx\boxed{220,259.84}$

If the prices go down by 10 percent, $P_{d}=130$K.

$M_{d}=\dfrac{P_{d}\cdot r^{m}_{15}}{1-\dfrac{1}{\left(1+r^{m}_{15}\right)^{180}}}=\dfrac{130,000\cdot 0.003}{1-\dfrac{1}{\left(1+\dfrac{3}{1000}\right)^{180}}}=\dfrac{130\cdot 3}{1-\dfrac{1}{\left(\dfrac{1003}{1000}\right)^{180}}}=\dfrac{390}{1-\left(\dfrac{1000}{1003}\right)^{180}}\approx\boxed{935.7442}$

$TP_{d}=180M_{d}\approx\boxed{168,433.99}$

$\mathbb{E} \left[TP\right]=0.45TP_{u}+0.55TP_{d}\approx0.45\times220,259.8+0.55\times168,434\approx\boxed{191755.62}$

\section*{4}

By \textbf{1}, we have the critical value $c^{*}\approx1410.1423$, comparing buying now and buying a year afterward.

Assume you are in the end of year 1. You are considering whether to buy in the end of $t=1$ or the end of $t=2$.

$P^{t=2}=\$200,000-60,000=\$140$K.

On the other hand,

$M^{t=2}_{15}=\dfrac{P^{t=2}\cdot r^{m}_{15}}{1-\dfrac{1}{\left(1+r^{m}_{15}\right)^{180}}}=\dfrac{140,000\cdot 0.003}{1-\dfrac{1}{\left(1+0.003\right)^{180}}}=\dfrac{140\cdot 3}{1-\dfrac{1}{\left(1.003\right)^{180}}}=\dfrac{420}{1-\dfrac{1}{\left(1.003\right)^{180}}}\approx1007.7248$

$TP_{2}=180M^{t=2}_{15}\approx181,390.46$

The critical value, comparing $t=1$ and $t=2$, is:

$c^{*\prime}=\dfrac{\rho^{m}}{1+\rho^{m}}\dfrac{TP_{1}-\dfrac{TP_{2}}{1+\rho^{a}}}{1-\dfrac{1}{\left(1+\rho_{m}\right)^{12}}}=\dfrac{600}{601}\dfrac{194,346-\dfrac{181,390.5}{1.02}}{1-\left(\dfrac{600}{601}\right)^{12}}\approx\boxed{1388.6526}$

If the rent at the end of $t=1$ is higher than $\$1388.6526$, buy at the end of $t=1$ (if you have already been in $t=1$). If the rent at the end of $t=1$ is less than $\$1388.6526$, wait another year and buy at the end of $t=2$ (if you have already been in $t=1$). 

\newpage

(i)

If $TP_{0}-C\left(c_{0}\right)-0.9c_{0}\left(\dfrac{1}{\left(1+\rho^{m}\right)^{12}}+\dfrac{1}{\left(1+\rho^{m}\right)^{13}}+\dots+\dfrac{1}{\left(1+\rho^{m}\right)^{23}}\right)-\dfrac{TP_{2}}{\left(1+\rho^{a}\right)^{2}}<0$, 

better to buy at $t=0$ than $t=2$ regardless of future change of rent.

That is,

\begin{multline*}
    207,303-c_{0}\dfrac{1+\rho^{m}}{\rho^{m}}\left(1-\dfrac{1}{\left(1+\rho^{m}\right)^{12}}\right)\\
    -0.9c_{0}\dfrac{1}{\left(1+\rho^{m}\right)^{12}}\left(1+\dfrac{1}{1+\rho^{m}}+\dots+\dfrac{1}{\left(1+\rho^{m}\right)^{11}}\right)-\dfrac{181,390.46}{\left(1+\rho^{a}\right)^{2}}<0
\end{multline*}

\begin{multline*}
    \implies207,303-c_{0}\dfrac{1+\rho^{m}}{\rho^{m}}\left(1-\dfrac{1}{\left(1+\rho^{m}\right)^{12}}\right)\\
    -\dfrac{0.9c_{0}}{\left(1+\rho^{m}\right)^{12}}\dfrac{1+\rho^{m}}{\rho^{m}}\left(1-\dfrac{1}{\left(1+\rho^{m}\right)^{12}}\right)-\dfrac{181,390.46}{\left(1+\rho^{a}\right)^{2}}<0
\end{multline*}

$\implies207,303-c_{0}\dfrac{1+\rho^{m}}{\rho^{m}}\left(1-\dfrac{1}{\left(1+\rho^{m}\right)^{12}}\right)\left(1+\dfrac{0.9}{\left(1+\rho^{m}\right)^{12}}\right)-\dfrac{181,390.46}{\left(1+\rho^{a}\right)^{2}}<0$

$\implies207,303-c_{0}\dfrac{1+\dfrac{1}{600}}{\dfrac{1}{600}}\left(1-\dfrac{1}{\left(1+\dfrac{1}{600}\right)^{12}}\right)\left(1+\dfrac{0.9}{\left(1+\dfrac{1}{600}\right)^{12}}\right)-\dfrac{181,390.46}{\left(1+0.02\right)^{2}}<0$

$\implies207,303-c_{0}\dfrac{\dfrac{601}{600}}{\dfrac{1}{600}}\left(1-\dfrac{1}{\left(\dfrac{601}{600}\right)^{12}}\right)\left(1+\dfrac{0.9}{\left(\dfrac{601}{600}\right)^{12}}\right)-\dfrac{181,390.46}{1.0404}<0$

$\implies207,303-601c_{0}\left(1-\left(\dfrac{600}{601}\right)^{12}\right)\left(1+0.9\left(\dfrac{600}{601}\right)^{12}\right)-\dfrac{181,390.46}{1.0404}<0$

$\implies207,303-\dfrac{181,390.46}{1.0404}<601c_{0}\left(1-\left(\dfrac{600}{601}\right)^{12}\right)\left(1+0.9\left(\dfrac{600}{601}\right)^{12}\right)$

$\implies601c_{0}\left(1-\left(\dfrac{600}{601}\right)^{12}\right)\left(1+0.9\left(\dfrac{600}{601}\right)^{12}\right)>207,303-\dfrac{181,390.46}{1.0404}$

$\implies c_{0}>\dfrac{207,303-\dfrac{181,390.46}{1.0404}}{601\left(1-\left(\dfrac{600}{601}\right)^{12}\right)\left(1+0.9\left(\dfrac{600}{601}\right)^{12}\right)}\approx\boxed{1472.52}>1410.1423$

Buy at $t=0$ if $c_{0}>1472.52$ regardless of future change.

If $TP_{0}-C\left(c_{0}\right)-1.1c_{0}\left(\dfrac{1}{\left(1+\rho^{m}\right)^{12}}+\dfrac{1}{\left(1+\rho^{m}\right)^{13}}+\dots+\dfrac{1}{\left(1+\rho^{m}\right)^{23}}\right)-\dfrac{TP_{2}}{\left(1+\rho^{a}\right)^{2}}>0$, 

better to buy at $t=2$ than $t=0$ regardless of future change of rent.

That is,

$\implies c_{0}<\dfrac{207,303-\dfrac{181,390.46}{1.0404}}{601\left(1-\left(\dfrac{600}{601}\right)^{12}\right)\left(1+1.1\left(\dfrac{600}{601}\right)^{12}\right)}\approx\boxed{1333.62}<1410.1423$

Buying at $t=0$ is the wrost.

Now let's consider:

$1.1\times 1333.62\approx1466.9779>1388.6526$

$0.9\times 1333.62<1333.62<1388.6526$

Still cannot tell $t=1$ and $t=2$.

If $c_{1}=1.1c_{0}<1388.6526\iff c_{0}<\boxed{1262.4115}<1333.62<1410.1423$, $t=2$ is better than $t=1$, $t=1$ is better than $t=0$,even if future increase in rate. Thus buying at $t=2$ is the best regardless of future rent change.

If $c_{1}=0.9c_{0}>1388.6526\iff c_{0}>\boxed{1542.9474}>1410.1423>1333.62$, buying at $t=2$ is wrose than buying at $t=1$ regardless of future change. But at the same time, $t=0$ becomes the best choice.

If $t=1$ is the best time,

$c_{0}<1410.1423$ and $c_{1}>1388.6526$.

$\because c_{1}\geqslant0.9c_{0}$

$\therefore0.9c_{0}>1388.6526$ regardless of rent change. Follows the conclusion above.

It follows that:

Buy at $t=0$ if $c_{0}>1472.52$ regardless of future change.

Buy at $t=2$ if $c_{0}<1262.4115$ regardless of future change.

But it's still hard to draw a conslusion if $1262.4115<c_{0}<1472.52$, unless you know exactly how rate will change.

\newpage

(ii)

Your expectation of the rent the next year is 

$\mathbb{E} \left[c_{1}\right]=0.5\cdot1.1c_{0}+0.5\cdot0.9c_{0}=c_{0}$

If $c_{0}>1410.1423>1388.6526$, buy the house at $t=0$.

If $c_{0}<1388.6526<1410.1423$, buy the house at $t=2$.

If $1388.6526<c_{0}<1410.1423$,

Buying at $t=1$ is better than buying at $t=0$, at the same time, $t=1$ is better than $t=2$.

So buying at $t=1$ is the optimal choice if $1388.6526<c_{0}<1410.1423$.

\underline{To sum up,}

If $c_{0}>1410.1423$, buy the house at $t=0$.

If $c_{0}=1410.1423$, indifferent between $t=0$ and $t=1$.

If $1388.6526<c_{0}<1410.1423$, buying at $t=1$ is the optimal choice.

If $c_{0}=1388.6526$, indifferent between $t=1$ and $t=2$.

If $c_{0}<1388.6526$, buy the house at $t=2$.

(iii)

If $1262.4115<c_{0}<1472.52$:

Case 1: If rent goes up at the end of $t=1$, $c_{1}=1.1c_{0}>1.1\times1262.4115\approx1388.6526$, buying a house at the end of $t=1$ is better than $t=2$.

If $c_{0}<1410.1423$, buying at $t=1$ is the best choice.

If $c_{0}>1410.1423$, buying at $t=0$ is the best choice.

Case 2: If rent goes down at the end of $t=1$, $c_{1}=0.9c_{0}<0.9\times1472.52\approx1325.27<1388.6526$, buying a house at the end of $t=2$ is better than buying a house at the end of $t=1$.

Since at $t=0$, you don't know the rent at $t=1$,

if $TP_{0}-C\left(c_{0}\right)-\mathbb{E} \left[c_{1}\right]\left(\dfrac{1}{\left(1+\rho^{m}\right)^{12}}+\dfrac{1}{\left(1+\rho^{m}\right)^{13}}+\dots+\dfrac{1}{\left(1+\rho^{m}\right)^{23}}\right)-\dfrac{TP_{2}}{\left(1+\rho^{a}\right)^{2}}<0$, 

buy at $t=0$ according to your expectation, $\mathbb{E} \left[c_{1}\right]=c_{0}$.

That implies, 

$c_{0}>\dfrac{207,303-\dfrac{181,390.46}{1.0404}}{601\left(1-\left(\dfrac{600}{601}\right)^{12}\right)\left(1+\left(\dfrac{600}{601}\right)^{12}\right)}\approx\boxed{1399.6309}$

If $c_{0}>1399.6309$, buying at $t=0$ is better than $t=2$.

If $c_{0}<1399.6309$, buying at $t=2$ is better than $t=0$.

If $1410.1423<c_{0}<1472.52$:

Buying at $t=0$ is better than buying at $t=1$, at the same time, $t=0$ is better than $t=2$.

If $1399.6309<c_{0}<1410.1423$:

Buying at $t=1$ is better than buying at $t=0$, at the same time, $t=0$ is better than $t=2$.

At the same time, $c_{1}=0.9c_{0}<0.9\times1410.1423\approx1269.1281<1388.6526$. $t=2$ is actually better than $t=1$.

The case is that, you know $t=1$ is better than buying at $t=0$ at the beginning. You wait to $t=1$. You find that rent goes down. $t=2$ becomes the best choice. But you cannot know the rent change in advance. At $t=0$, your expectation is that $t=1$ is the best.

If $1262.4115<c_{0}<1399.6309$:

Buying at $t=2$ is better than buying at $t=0$, at the same time, 

$c_{1}=0.9c_{0}<0.9\times1399.6309\approx1259.6662<1388.6526$

$t=2$ is better than $t=1$,if rent goes down.

\underline{In conclusion,}

If $c_{0}>1472.52$, buy at $t=0$.

If $c_{0}<1262.4115$, buy at $t=2$.

If $1262.4115\leqslant c_{0}\leqslant1472.52$:

Case 1: rent increases.

If $c_{0}=1262.4115$, indifferent with $t=1$ and $t=2$.

If $1262.4115<c_{0}<1410.1423$, buy at $t=1$.

If $c_{0}=1410.1423$, indifferent with $t=0$ and $t=1$.

If $1410.1423<c_{0}\leqslant1472.52$, buy at $t=0$.

Case 2: rent decreases,

If $1410.1423<c_{0}\leqslant1472.52$, best choice is to buy at $t=0$.

If $1262.4115\leqslant c_{0}<1410.1423$, best choice is to buy at $t=2$.

(iv)

Follows from (iii), but plug in $0.9c_{0}$ in case 2.

It implies $c_{0}>\dfrac{207,303-\dfrac{181,390.46}{1.0404}}{601\left(1-\left(\dfrac{600}{601}\right)^{12}\right)\left(1+0.9\left(\dfrac{600}{601}\right)^{12}\right)}\approx1472.52$

If $c_{0}>1472.52$, buying at $t=0$ is the best.

If $c_{0}<1472.52$, buying at $t=2$ is better than $t=0$ if price goes down.

Case 2 is under the condition $1262.4115<c_{0}<1472.52$, where buying a house at the end of $t=2$ is better than buying a house at the end of $t=1$ if rent goes down. 

So in Case 2, rent goes down, buy at $t=2$ is optimal.

\underline{To summarize:} 

If $c_{0}>1472.52$, buy at $t=0$.

If $c_{0}<1262.4115$, buy at $t=2$.

If $1262.4115\leqslant c_{0}\leqslant1472.52$:

Case 1, rent goes up, 

If $c_{0}=1262.4115$, indifferent with $t=1$ and $t=2$

if $1262.4115<c_{0}<1410.1423$, buy at $t=1$,

if $c_{0}=1410.1423$, indifferent with $t=0$ and $t=1$,

if $1410.1423<c_{0}\leqslant1472.52$, buy at $t=0$,

Case 2, rent goes down, buy at $t=2$.

(v)

If $1262.4115<c_{0}<1472.52$:

When $1410.1423<c_{0}<1472.52$, $t=0$ is better than $t=1$. You are comparing $t=0$ and $t=2$.

Plug in $\mathbb{E} \left[c_{1}\right]$ like (iii), we have $1399.6309<1410.1423<c_{0}<1472.52$. $t=0$ is better than $t=2$.

When $1262.4115<c_{0}<1410.1423$, $t=1$ is better than $t=0$. You wait for another year.

If price goes up, $c_{1}=1.1c_{0}>1.1\times1262.4115\approx1388.6526$, you buy at $t=1$.

If price goes down, $c_{1}=0.9c_{0}<0.9\times1410.1423\approx1269.1281<1388.6526$, you buy at $t=2$.

\fbox{%
  \parbox{\textwidth}{
    In short,\\
    \\
    If $c_{0}>1410.1423$, buy at $t=0$.\\
    \\
    If $c_{0}<1262.4115$, buy at $t=2$.\\
    \\
    If $1262.4115\leqslant c_{0}\leqslant1410.1423$: \\
    \\
    Wait for one year.\\
    \\
    When rent goes up, buy at $t=1$.\\
    \\
    When rent goes down, buy at $t=2$.\\    
  }%
}

(vi)

If $1262.4115<c_{0}<1472.52$:

When $1410.1423<c_{0}<1472.52$, $t=0$ is better than $t=1$. You are comparing $t=0$ and $t=2$.

If rent goes up, $c_{0}>1333.62$, buy at $t=0$.

If rent goes down, $c_{0}<1472.52$, buy at $t=2$.

When $1262.4115<c_{0}<1410.1423$:

Wait for one year.

If rent goes up, $c_{1}=1.1c_{0}>1.1\times1262.4115\approx1388.6526$, you buy at $t=1$.

If price goes down, $c_{1}=1.1c_{0}<0.9\times1410.1423\approx1269.1281<1388.6526$, you buy at $t=2$.

\underline{Thus,}

If $c_{0}>1472.52$, buy at $t=0$.

If $c_{0}<1262.4115$, buy at $t=2$.

If $1410.1423<c_{0}<1472.52$:

If rent goes up, buy at $t=0$.

If rent goes down, buy at $t=2$.

If $1262.4115\leqslant c_{0}\leqslant1410.1423$:

Wait for one year.

When rent goes up, buy at $t=1$.

When rent goes down, buy at $t=2$.
\\
\\
I think (v) is closest to professor's meaning.

\end{document}